%%%%%%%%%%%%%%%%%%%%%%%%%%%%%%%%%%%%%%%%%
% Short Sectioned Assignment
% LaTeX Template
% Version 1.0 (5/5/12)
%
% This template has been downloaded from:
% http://www.LaTeXTemplates.com
%
% Original author:
% Frits Wenneker (http://www.howtotex.com)
% % License:
% CC BY-NC-SA 3.0 (http://creativecommons.org/licenses/by-nc-sa/3.0/)
%
%%%%%%%%%%%%%%%%%%%%%%%%%%%%%%%%%%%%%%%%%

%----------------------------------------------------------------------------------------
%	PACKAGES AND OTHER DOCUMENT CONFIGURATIONS
%----------------------------------------------------------------------------------------

\documentclass[paper=a4, fontsize=11pt]{scrartcl} % A4 paper and 11pt font size

\usepackage[T1]{fontenc} % Use 8-bit encoding that has 256 glyphs
\usepackage{fourier} % Use the Adobe Utopia font for the document - comment this line to return to the LaTeX default
\usepackage[english]{babel} % English language/hyphenation
\usepackage{amsmath,amsfonts,amsthm} % Math packages

\usepackage{minted} % Allows to put our code :)
\usepackage{graphicx} % Allows to put images :)
\usepackage[usenames, dvipsnames]{color} % Allows to have color :)
\usepackage{tikz} % Used for drawing state machines
\usepackage{pgf} % Used for drawing state machines
\usetikzlibrary{automata, positioning}
\usetikzlibrary{arrows}

\usepackage{sectsty} % Allows customizing section commands
\allsectionsfont{\centering \normalfont\scshape} % Make all sections centered, the default font and small caps

\usepackage{fancyhdr} % Custom headers and footers
\pagestyle{fancyplain} % Makes all pages in the document conform to the custom headers and footers
\fancyhead{} % No page header - if you want one, create it in the same way as the footers below
\fancyfoot[L]{} % Empty left footer
\fancyfoot[C]{} % Empty center footer
\fancyfoot[R]{\thepage} % Page numbering for right footer
\renewcommand{\headrulewidth}{0pt} % Remove header underlines
\renewcommand{\footrulewidth}{0pt} % Remove footer underlines
\setlength{\headheight}{13.6pt} % Customize the height of the header

\numberwithin{equation}{section} % Number equations within sections (i.e. 1.1, 1.2, 2.1, 2.2 instead of 1, 2, 3, 4)
\numberwithin{figure}{section} % Number figures within sections (i.e. 1.1, 1.2, 2.1, 2.2 instead of 1, 2, 3, 4)
\numberwithin{table}{section} % Number tables within sections (i.e. 1.1, 1.2, 2.1, 2.2 instead of 1, 2, 3, 4)

\setlength\parindent{0pt} % Removes all indentation from paragraphs - comment this line for an assignment with lots of text

%----------------------------------------------------------------------------------------
%	TITLE SECTION
%----------------------------------------------------------------------------------------

\newcommand{\horrule}[1]{\rule{\linewidth}{#1}} % Create horizontal rule command with 1 argument of height

\title{
\normalfont \normalsize
\textit{In The Name of God} \\ \textsc{Computer Engineering Department of Amirkabir University of Technology} \\ [25pt] \horrule{0.5pt} \\[0.4cm] % Thin top horizontal rule
\huge Microprocessors Homework 8  \\ % The assignment title
\horrule{2pt} \\[0.5cm] % Thick bottom horizontal rule
}

\author{Iman Tabrizian (9331032)}

\date{\normalsize\today}

\begin{document}

\maketitle

\section{Question 1}
They have different structures inherently that they can't be used interchangably.
We explain some of the characteristics of Latches:

Latches have the following attributes and functionalities:

\begin{enumerate}
    \item
        Outputs need to be consistent over a time period. So latches use \textbf{D-FF}
        to implement this characteristic.
    \item
        In order to capture input and persist output latches need to use a control signal.
\end{enumerate}


Buffers have the following attributes and functionalities:


\begin{enumerate}
    \item
        Buffers are used to guarantee the power of the signal and to further
        amplification if required.

    \item
        They are used to make the output signal robust.

\end{enumerate}

As you can see they are different and have different use cases. So they
can't be used interchangebly.

\section{Question 2}
In polling you keep checking for the event you want to occur. But in the
interrupt mode you should only do the job you want when the event occurs you'll
be informed about it and the program stops resuming and starts executing the
interrupt routine. In this mode you don't have to constantly poll for the event
to occur.



\section{Question 3}
Yes, you can. By first polling the device with more priority and then checking
for the other you can apply priority to output devices.

\section{Question 4}
The only register that Atmega16 stores is the current address. It only stores PC.
\end{document}
